\chapter{Ethical Issues}

This section aims to briefly outline potential ethical concerns associated with this project.
There are three main domains to consider for this project; health and safety including potential human risks, the possibility of military utilisation and the energy requirements for training.

Despite these precautionary measures significantly reducing the likelihood of human injury, the aspect of safety remains a pivotal consideration throughout the project's duration.
Given the project's involvement with autonomous aerial vehicles, there is an inherent risk of human harm during the testing process.
To mitigate the risk of and ensure precise measurements during tests, all demonstrations and evaluations will be done entirely in an indoor environment, namely the aerial robotics lab.
Health and safety checks, along with risk assessments will need to be carried out before usage.
Furthermore, all autonomous flight tasks will initially be performed in a simulated environment in advance of being carried out.
These precautionary measures significantly reduce the likelihood of human injury, but the safety aspect will be kept in mind throughout the process.

The main ethical concern of this project is the potential for military applications.
Since UAVs are extensively employed in military contexts, this raises the concern about the possible misuse of this research.
However, the scope of application for this work is inherently limited to military use.
This work is motivated by environmental sensing over extended periods, a requirement that is not as aligned with military operations.
Additionally, this project focuses on using a cost-effective solution, namely a tethered drone, which, despite its economic nature, is not designed for stealth or discreet operations.
The perching manoeuvres would be unlikely to have a military purpose.
Nevertheless, it's essential to acknowledge that the field of Reinforcement Learning for UAVs could be adapted for use in these applications.

The final ethical issue to consider is the energy usage in training.
The training will likely require significant energy usage.
However, a primary objective of this project is to reduce the extent of training required and to maximize the efficiency of simulation usage
This should have the added benefit of also keeping energy usage lower.
